\documentclass{article}
\usepackage{graphicx} % Required for inserting images
\usepackage[T1,T2A]{fontenc}
\usepackage[utf8]{inputenc}
\usepackage[russian,english]{babel}
\usepackage{hyperref}
% \usepackage{csquotes}


% \usepackage{python_pack}
\usepackage{minted}
\usemintedstyle{emacs}
\usepackage{xcolor} % to access the named colour LightGray
\definecolor{LightGray}{gray}{0.9}

\usepackage{geometry} % Простой способ задавать поля
\geometry{top=20mm}
\geometry{bottom=25mm}
\geometry{left=30mm}
\geometry{right=10mm}

\usepackage{setspace} % Интерлиньяж
\setstretch{1.1}

\newcommand{\coursename}{Causal Inference: прозрение и практика}

\usepackage{amsmath,amssymb}
\DeclareMathOperator{\E}{\mathbb{E}}

\title{
    \textbf{\coursename}\\
    Лекция 1.
    Основные понятия Causal Inference
    }
\author{Юрашку Иван Вячеславович}
\date{\today}

\begin{document}

    % \begin{center}
    %     \textbf
    % \end{center}

    \maketitle

    \section*{Что такое Causal Inference.}

        В современном мире Data Science играет ключевую роль в анализе и использовании данных. Мы живем в эпоху, где информация - это золото, а умение извлекать из нее ценные знания - наша сила. Однако часто понятие Data Science ограничивается лишь алгоритмами машинного обучения или даже искусственным интеллектом, умаляя другие важные аспекты этой дисциплины.

        Вот где начинается история о сближении двух мощных инструментов: эконометрики и Machine Learning. В разных эпохах они казались как бы двумя противоположными полярностями в анализе данных. Машинное обучение стремилось к высокой точности прогнозов, зачастую уступая интерпретируемости моделей. С другой стороны, эконометрика ставила акцент на интерпретируемость, понимание причинно-следственных связей, иногда уходя в тень из-за ограниченности моделей.

        Однако со временем стало понятно, что для полного понимания данных нам нужно объединить эти подходы. И здесь на сцену выходит Causal Inference, или причинно-следственная связь. Этот инструмент помогает нам разгадывать причины за явлениями, объединяя преимущества как машинного обучения, так и эконометрики. Так, Judea Pearl в \href{https://www.degruyter.com/document/doi/10.1515/jci-2021-0006/html}{своей статье} 2021 года подчеркивает важность CI как ключевого элемента для достижения баланса между эмпирическим и интерпретируемым.

        \begin{figure}[h]
            \centering
            \includegraphics[width=0.7\linewidth]{imgs/epic_CI.jpg}
            % \caption{meme}
            \label{fig:mpr}
        \end{figure}

        \newpage

        Погружение в мир причинно-следственной связи и машинного обучения не только расширит ваш кругозор, но и даст вам ключ к разгадке сложных и важных вопросов, стоящих перед современным обществом.

        Допустим, вы владеете интернет-магазином и хотите понять, какие факторы влияют на продажи. С помощью методов причинно-следственной связи вы сможете определить, какие из ваших маркетинговых кампаний действительно приносят наибольший доход, и направить свои усилия и ресурсы в нужное русло.

        Еще один пример - медицинская сфера. С помощью анализа причинно-следственных связей можно выявить, какие лечебные методы наиболее эффективны для конкретного заболевания, что позволит разрабатывать более точные и эффективные методики лечения.

        Этот курс - не просто набор теории, он предлагает вам практические инструменты для анализа данных и принятия обоснованных решений. С его помощью вы сможете выйти за рамки обычных аналитических методов и раскрыть потенциал данных, лежащих у вас под рукой. Полученные знания не только помогут читателю в работе, но и дадут возможность вносить реальные изменения в мир вокруг нас.

    \section*{И все же Causal Inference - это не ML}
        Машинное обучение в настоящее время успешно решает задачи прогнозирования. Как подчеркивают Ajay Agrawal, Joshua Gans и Avi Goldfarb в книге "Prediction Machines":

        \begin{quote}
            "Новая волна искусственного интеллекта на самом деле приносит нам не интеллект, а важный компонент интеллекта - прогнозирование".
        \end{quote}

        С машинным обучением мы можем совершать самые разнообразные и впечатляющие вещи. Главное требование заключается в том, чтобы сформулировать наши задачи как задачи прогнозирования. Хотите перевести текст с английского на португальский? Тогда создайте модель машинного обучения, которая предсказывает португальские предложения по английским. Хотите распознавать лица? Тогда разработайте модель машинного обучения, которая определяет наличие лица в определенной области изображения. Хотите создать автомобиль с автоматическим управлением? Тогда создайте модель машинного обучения, которая предсказывает направление поворота руля, а также давление на тормоза и акселератор при предоставлении изображений и данных с сенсоров, полученных из окружающей среды автомобиля.


        Однако ML - не панацея. Он может производить чудеса в рамках строгих условий, но при этом может потерпеть крах, если данные немного отличаются от того, что модель привыкла видеть.


        Машинное обучение известно своей неспособностью решать проблемы обратной причинности. Оно требует ответа на вопросы типа "а что, если"{}, которые экономисты называют контрфактуальными. Как отмечается в "Prediction Machines"{}, ML не справляется с такими задачами. Оно может предсказывать на основе данных, но не может оценить воздействие изменений. В качестве примера из книги "Prediction Machines":

        \begin{quote}
            "Во многих отраслях низкая цена ассоциируется с низкими продажами. Например, в гостиничной индустрии цены низки вне туристического сезона, а в период пикового спроса цены высоки и гостиницы полностью заполнены. Исходя из этих данных, наивное предположение может подсказать, что повышение цены приведет к увеличению числа проданных номеров".
        \end{quote}

        По сути, ответ на вопросы о причинности является более сложной задачей, чем многие могут подумать. Это то, чему посвящен курс "\coursename". В нем мы исследуем, как использовать данные для изучения причинно-следственных связей и оценки воздействия вмешательств на результаты. Поехали!

    \section*{Постановка задачи и обозначения}

        Определимся с тем, что конкретно мы хотим научиться делать. Формализуем задачу следующим образом.

        Пусть существует множество объектов. Это может быть пациент, потенциальный клиент коммерческой компании, город, что угодно. Значение параметра одного конкретного объекта будем обозначать с добавлением верхнего индекса, а вектор значений, соответствующий этому параметру - будет употребляться без индекса: $X^i,\; X$.

        Пусть существует возможность подействовать на объект. В реальном мире это может быть лечение пациента, рекламная кампания, введение юридических ограничений или правил того, как вести себя в людных местах. Ограничений практически нет. Оказанное влияние будем рассматривать как двоичный признак - True или False, без учета возможной интенсивности влияния.

        В терминологии causal inference воздействие, которое исследуется или рассматривается, называется "treatment"{} (лечение) из-за того, что этот термин часто используется в контексте медицинских испытаний, где исследуют воздействие определенного лечебного метода или медикамента на пациентов. Однако в более широком смысле "treatment"{} может означать любое воздействие или изменение, которое предпринимается с целью изучения его эффекта на исследуемую систему объектов. Следуя прижившимся нормам, обозначим этот параметр словом treatment (лечение), или, кое-где в формулах для краткости просто буквой T. Формально, когда мы говорим о воздействии - это всего лишь разделение группы объектов на две части по некоторому бинарному признаку. Фактически, какого-либо значимого воздействия может и не быть. В таком случае мы имеет дело с фиктивным воздействием - часто встречаемый случай.

        Целевую переменную изучаемого объекта, она же "$target$"{}, будем обозначать как $Y$. Обычно это вещественная величина. Договоримся сразу, дабы избежать путаницы, что $Y$ - это показатель, измеряемый в конкретный промежуток времени, часто - в еще не наступивший. Например, при исследовании смс-оповещений, нас может интересовать, каким образом это отразится через четыре месяца на продаваемости определенного продукта. При этом данный показатель в текущий момент времени назовем лаговым значением целевой переменной. Когда он будет нам попадаться - будем обозначать его отдельным символом, например $Y_\texttt{lag 4 month}$.

        Теперь представим, что для каждого изучаемого объекта существует две вселенных, отличающихся одним только наличием воздействия на него. Пусть также мы могли бы узнать значения целевой переменной как при $T=1$, так и при $T=0$. Обозначим эти величины за $Y_{i0} = Y_i|_{T=0}$ и  $Y_{i1}=Y_i|_{T=1}$.

        Их разность обозначается как "treatment effect" ($TE_i=Y_{i1}-Y_{i0}$). Эта величина - реальное отражение эффекта между реальностью

        Тогда наш истинный интерес заключается в том, чтобы оценить величину \textbf{среднего эффекта от воздействия}, или average treatment effect:
        $$
            ATE=
            \frac{1}{n}\displaystyle\sum_{i=1}^nTE_i=
            \E{}TE=
            \E{}(Y_1-Y_0).
        $$
        На данном этапе легко запутаться, но ни в коем случае мы не должны путать между собой две следующих величины:
        \begin{itemize}
          \item $\E{}(Y_1-Y_0) = \E{}(Y|_{T=1}-Y|_{T=0})$ -- средняя разность ...
          \item $\E{}(Y|T=1) - \E{}(Y|T=0)$
        \end{itemize}

        Чтобы получше разобраться, рассмотрим пример.

        \begin{minted}
        [
        frame=lines,
        framesep=2mm,
        baselinestretch=1.2,
        bgcolor=LightGray,
        fontsize=\footnotesize,
        linenos
        ]
        {python}
        import numpy as np

        pass
        \end{minted}





    \section*{Correlation is not causation}

        "Все события вымышлены, любые совпадения случайны" - приблизительно так обязывают нас правила этики начать данный параграф.

        Итак, в одной стране, в одном крупном городе, существовал госпиталь для лечения определенного рода больных. Через дорогу от него решили возвести абсолютно новый медицинский центр, специализирующийся на том же смертельном недуге. Он был оснащен всеми передовыми технологиями и туда пригласили работать самых лучших докторов этой области медицины. Целью было внедрить новые методы хирургии. Поскольку любой такой новый дорогостоящий проект хотелось бы обмерять со всех сторон, было принято решение некоторое время позволить функционировать обеим организациям для того, чтобы выявить разницу в подходах к лечению на практике.

        Информация о новом кампусе разлетелась со скоростью звука, выдыхаемого друзьями или родственниками тяжело заболевших (это несколько другая скорость, заметно превышающая обычную скорость звука), и вскоре люди хлынули за возможностью получить необыкновенный шанс.

        Прошли месяцы, и результаты нового медицинского центра начали анализироваться. По первоначальным отчетам казалось, что он показывает более высокую смертность среди пациентов по сравнению с соседним госпиталем. Однако, при более детальном рассмотрении данных стало ясно, что существовал статистический bias в распределении пациентов между этими двумя учреждениями.

        Оказалось, что вновь открывшийся медицинский центр привлекал в основном пациентов с более тяжелыми формами заболевания, так как люди их семьи или родственники стремились обратиться туда за получением новейших методов лечения. В то время как госпиталь продолжал принимать пациентов с менее серьезными случаями.

        Этот bias в распределении пациентов искажал общую картину, делая исходные данные о смертности в медицинском центре недостоверными для сравнения с госпиталем. Таким образом, необходимо быть внимательным к подобным факторам при анализе результатов исследований, чтобы избежать искажений и сделать более точные выводы о качестве медицинского обслуживания.

        \begin{minted}
        [
        frame=lines,
        framesep=2mm,
        baselinestretch=1.2,
        bgcolor=LightGray,
        fontsize=\footnotesize,
        linenos
        ]
        {python}
        import numpy as np

        def incmatrix(genl1,genl2):
            m = len(genl1)
            n = len(genl2)
            M = None #to become the incidence matrix
            VT = np.zeros((n*m,1), int)  #dummy variable

            #compute the bitwise xor matrix
            M1 = bitxormatrix(genl1)
            M2 = np.triu(bitxormatrix(genl2),1)

            for i in range(m-1):
                for j in range(i+1, m):
                    [r,c] = np.where(M2 == M1[i,j])
                    for k in range(len(r)):
                        VT[(i)*n + r[k]] = 1;
                        VT[(i)*n + c[k]] = 1;
                        VT[(j)*n + r[k]] = 1;
                        VT[(j)*n + c[k]] = 1;

                        if M is None:
                            M = np.copy(VT)
                        else:
                            M = np.concatenate((M, VT), 1)

                        VT = np.zeros((n*m,1), int)

            return M
        \end{minted}

\end{document}
