\documentclass{article}
\usepackage{graphicx} % Для вставки изображений
\usepackage[T1,T2A]{fontenc}
\usepackage[utf8]{inputenc} % Кодировка UTF-8
\usepackage[russian,english]{babel} % Языковые настройки для русского и английского
\sloppy % Избегаем выхода за границы страницы

\usepackage{hyperref} % Для создания гиперссылок

\usepackage{fvextra} % Расширенные окружения для вербатима
\usepackage{csquotes} % Контекстно-чувствительные цитаты

\usepackage{listings} % Настройка отображения исходного кода
\lstset{
    basicstyle=\ttfamily,
    breaklines=true,
    showstringspaces=false,
    commentstyle=\color{gray},
    keywordstyle=\color{blue},
    frame=tb,
    captionpos=b
}

\usepackage{minted} % Подсветка синтаксиса для исходного кода
\usemintedstyle{friendly} % Стиль подсветки по умолчанию (можно выбрать другой)

\usepackage{xcolor} % Для доступа к именованным цветам
\definecolor{LightGray}{gray}{0.9} % Определение пользовательского цвета

\usepackage{geometry} % Простая настройка полей страницы
\geometry{
    top=20mm,
    bottom=25mm,
    left=30mm,
    right=10mm
}

\usepackage{setspace} % Настройка межстрочного интервала
\setstretch{1.1}

\usepackage{amsmath,amssymb} % Математические символы и окружения
\DeclareMathOperator{\E}{\mathbb{E}} % Определение оператора математического ожидания
