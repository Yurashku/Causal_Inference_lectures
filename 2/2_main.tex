\documentclass{article}
\usepackage{graphicx} % Required for inserting images
\usepackage[T1,T2A]{fontenc}
\usepackage[utf8]{inputenc}
\usepackage[russian,english]{babel}
\usepackage{hyperref} % For hyperlinks

\usepackage{fvextra} % Enhanced verbatim environments
\usepackage{csquotes} % Context sensitive quotation facilities

\usepackage{listings} % Typeset source code listings
\lstset{
    basicstyle=\ttfamily,
    breaklines=true,
    showstringspaces=false,
    commentstyle=\color{gray},
    keywordstyle=\color{blue},
    frame=tb,
    captionpos=b
}

\usepackage{minted} % Syntax highlighting for source code
\usemintedstyle{friendly} % Set minted style

\usepackage{xcolor} % To access named colours
\definecolor{LightGray}{gray}{0.9} % Define a custom color

\usepackage{geometry} % Simple package for setting the page layout
\geometry{
    top=20mm,
    bottom=25mm,
    left=30mm,
    right=10mm
}

\usepackage{setspace} % Set line spacing
\setstretch{1.1}

\usepackage{amsmath,amssymb} % Mathematical symbols and environments
\DeclareMathOperator{\E}{\mathbb{E}} % Define math operator for expectation

% Document information
\newcommand{\coursename}{Causal Inference: прозрение и практика}
\title{
    \textbf{\coursename}\\
    Лекция 2. Рандомизированные контролируемые испытания
}
\author{Юрашку Иван Вячеславович}
\date{\today}

\begin{document}

\maketitle

\section{Введение}

    В прошлой статье мы встретились с основным препятствием того, чтобы считать $АТЕ$ как разность средних между целевой и контрольной группами.
    Название этого препятствия, bias, переводится с английского как <<смещение>> либо <<предвзятость>>.
    Немного неродственных два значения, тем не менее оба хорошо подходят для описания ситуации.

    Действительно, если $BIAS(\mathcal{M})$ отличен от нуля, то отлично от нуля по крайней мере одно из значений BIAS(M0, M1, 0), BIAS(M0, M1, 1).

    Рандомизированные контролируемые испытания (РКИ) являются золотым стандартом для оценки эффектов вмешательств и определения причинно-следственных связей.
    Они помогают минимизировать предвзятость и обеспечивают объективное сравнение между группами.

\section{Постановка задачи РКИ}

    Целью РКИ является случайное распределение участников между группами treatment и control, что гарантирует равномерность всех потенциальных факторов, влияющих на исход.

\section{Сценарии проведения РКИ: AB тесты}

    AB тесты являются одним из наиболее распространенных сценариев РКИ. Они используются для оценки эффективности изменений на сайте или в продукте.

\subsection{Описание AB теста}

    AB тест состоит из двух групп: treatment (новая версия продукта) и control (текущая версия продукта). Участники случайным образом назначаются в одну из этих групп.

\subsection{Формулы}

    Формула для оценки среднего эффекта вмешательства (ATE):

    \[
    \text{ATE} = \E[Y(1) - Y(0)]
    \]

    Формула для BIAS(M0, M1, t):

    \[
    \text{BIAS}(M0, M1, t) = \E[M1(t) - M0]
    \]

\section{Примеры РКИ}

\subsection{Медицинское исследование}

Пример медицинского исследования с применением РКИ.

\subsection{AB тест в интернет-маркетинге}

Пример использования AB теста для оценки конверсий на сайте.

\section{Заключение}

РКИ играют ключевую роль в науке и практике, обеспечивая объективное оценивание эффектов вмешательств.

\end{document}
