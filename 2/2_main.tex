\documentclass{article}
\usepackage{graphicx} % Required for inserting images
\usepackage[T1,T2A]{fontenc}
\usepackage[utf8]{inputenc}
\usepackage[russian,english]{babel}
\usepackage{hyperref} % For hyperlinks

\usepackage{fvextra} % Enhanced verbatim environments
\usepackage{csquotes} % Context sensitive quotation facilities

\usepackage{listings} % Typeset source code listings
\lstset{
    basicstyle=\ttfamily,
    breaklines=true,
    showstringspaces=false,
    commentstyle=\color{gray},
    keywordstyle=\color{blue},
    frame=tb,
    captionpos=b
}

\usepackage{minted} % Syntax highlighting for source code
\usemintedstyle{friendly} % Set minted style

\usepackage{xcolor} % To access named colours
\definecolor{LightGray}{gray}{0.9} % Define a custom color

\usepackage{geometry} % Simple package for setting the page layout
\geometry{
    top=20mm,
    bottom=25mm,
    left=30mm,
    right=10mm
}

\usepackage{setspace} % Set line spacing
\setstretch{1.1}

\usepackage{amsmath,amssymb} % Mathematical symbols and environments
\DeclareMathOperator{\E}{\mathbb{E}} % Define math operator for expectation

% Document information
\newcommand{\coursename}{Causal Inference: прозрение и практика}
\title{
    \textbf{\coursename}\\
    Лекция 2. Рандомизированные контролируемые испытания
}
\author{Юрашку Иван Вячеславович}
\date{\today}

\begin{document}

    \maketitle

    \section{Введение}

    \textbf{Рандомизированные контролируемые испытания} (РКИ) представляют собой наиболее объективную, прозрачную и эффективную методологию для проведения экспериментов.
    Они пользуются огромной популярностью и применяются в самых разных сферах, включая науку, медицину, маркетинг и технологии.
    С их помощью учёные и специалисты могут проверять эффективность новых методов лечения, лекарственных препаратов, продуктов или услуг, сравнивая результаты между двумя или более группами.
    РКИ встречаются гораздо чаще, чем может показаться на первый взгляд.
    Это невероятно популярный метод исследования причинно-следственных связей.
    Хотя они довольно просты в реализации, их точность значительно превосходит все другие методы аппроксимации $ATE$.


    Существует несколько видов рандомизированных контролируемых исследований. Самые используемые из них:

    \begin{itemize}
        \item \textbf{Простая рандомизация} --- каждому участнику испытания случайным образом назначается либо исследуемое вмешательство, либо контрольное.
        \item \textbf{Стратифицированная рандомизация} --- участники сначала разделяются на страты на основе определённых характеристик, а затем внутри каждой страты происходит случайное распределение по группам исследования.
        \item \textbf{Кластерная рандомизация} --- в этом случае рандомизация происходит по группам или <<кластерам>>, а не по отдельным участникам.
        \item \textbf{Кроссоверное испытание} --- сначала участники получают одно вмешательство, а после определённого периода времени -- другое (и наоборот).
        \item \textbf{Факториальное испытание} --- каждый участник случайным образом распределяется по группе, которая получает определённую комбинацию вмешательств, включая плацебо.
    \end{itemize}

    \href{https://www.bmj.com/content/340/bmj.c723.full}{Анализ 616 РКИ, проиндексированных в PubMed в декабре 2006 года}, показал, что 78\% были исследованиями в формате простой рандомизации, 16\% были кроссоверными, 2\% были стратифицированными, 2\% были кластерными и 2\% были факториальными.


    \section{A/B тест}

        В контексте решения поставленной задачи по оцениванию $ATE$ при использовании двух групп, научимся применять РКИ первого типа.
        А именно РКИ, предполагающие наличие двух групп, контрольной и целевой, обозначаемых как A и B, из-за чего также именуемые как A/B тесты.

        В предыдущей статье мы столкнулись с ключевым препятствием для рассмотрения $ATE$ как разницы средних значений между целевой и контрольной группами.
        Это явление называется bias, что в переводе с английского означает <<смещение>> или <<предвзятость>>.
        Несмотря на их кажущуюся неродственность, вместе эти два термина хорошо описывают ситуацию.

        Действительно, согласно выводу из предыдущей главы, если $BIAS(M)$ не равен нулю, то по крайней мере одно из значений $BIAS(\mathcal{M}, 0)$ или $BIAS(\mathcal{M}, 1)$ также не равно нулю.
        Предположим, что $BIAS(\mathcal{M}, 0) \neq 0$.

        \[
            BIAS(\mathcal{M}, 0) =
            \frac{1}{|\mathcal{M}{(1)}|} \displaystyle\sum{i\in\mathcal{M}_{(1)}}
            Y^i_{(0)} -
            \frac{1}{|\mathcal{M}{(0)}|} \displaystyle\sum{i\in\mathcal{M}_{(0)}}
            Y^i_{(0)} \neq 0
        \]

        Иначе говоря, это означает, что ещё до начала эксперимента эти две группы не были одинаковыми.
        Если бы наш эксперимент не проводился, то обе группы не подвергались бы воздействию, и их средние значения целевой переменной всё равно не были бы равны друг другу.
        Это и есть предвзятость.

        Чтобы преодолеть эту предвзятость, как один из способов, эксперты в области причинно-следственного вывода проводят рандомизированные контролируемые испытания (РКИ).
    \section{Сценарии проведения РКИ: AB тесты}

        AB тесты являются одним из наиболее распространенных сценариев РКИ. Они используются для оценки эффективности изменений на сайте или в продукте.

    \subsection{Описание AB теста}

        AB тест состоит из двух групп: treatment (новая версия продукта) и control (текущая версия продукта). Участники случайным образом назначаются в одну из этих групп.

    \subsection{Формулы}

        Формула для оценки среднего эффекта вмешательства (ATE):

        \[
        \text{ATE} = \E[Y(1) - Y(0)]
        \]

        Формула для BIAS(M0, M1, t):

        \[
        \text{BIAS}(M0, M1, t) = \E[M1(t) - M0]
        \]

    \section{Примеры РКИ}

    \subsection{Медицинское исследование}

    Пример медицинского исследования с применением РКИ.

    \subsection{AB тест в интернет-маркетинге}

    Пример использования AB теста для оценки конверсий на сайте.

    \section{Заключение}

    РКИ играют ключевую роль в науке и практике, обеспечивая объективное оценивание эффектов вмешательств.

\end{document}
